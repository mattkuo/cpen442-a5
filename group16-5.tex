\documentclass[conference]{IEEEtran}

\usepackage{biblatex}
\usepackage{listings}
\usepackage{amsmath}
\usepackage{url}
\usepackage{graphicx}


\bibliography{refs.bib}

\begin{document}

\title{An Analysis of the Ventir OSX Keylogger}


% author names and affiliations
% use a multiple column layout for up to three different
% affiliations
\author{\IEEEauthorblockN{blue, v6q8}
\IEEEauthorblockA{Department of Computer Science\\
University of British Columbia\\
}
\and
\IEEEauthorblockN{Matt Labbe, hydro}
\IEEEauthorblockA{Electrical and\\Computer Engineering\\
University of British Columbia\\
}
}

\maketitle

\begin{abstract}
This article seeks to analyze the behavior and mechanisms of the Ventir malware.
Prevention measures and security principles are also topics that are
investigated in this study.
\end{abstract}

\section{Introduction}
Throughout past decade, there has been an extensive surge in public concern
concerning information privacy. The advancements of technology including the
smartphone and the internet along with public disclosure of govermental and
commercial breaches of private information has brought along a mass public
interest in information security\cite{eset_trends}.

Keyloggers, malicious software which are used to record victims' keystrokes, are
an area of interest in regards to information security breaches. This study
focuses on the Ventir backdoor keylogger which was discovered in October 2014. A
keylogger that only runs on Mac OSX, the Ventir malware is a suspect of recent
leaks of login and passwords from some major e-mail
providers\cite{kuzin_ventir_2014}.

\section{Propagation \& Target Selection}
At the time of writing, security experts are unsure of how the malware is spread
although it is suspected that pirated software from peer-to-peer networks could
be to blame\cite{kuzin_ventir_2014}. As the malware only runs on Mac OSX which
is catered towards a wealthier demographic, it is conceivable that malware
creators looking for higher financial gains would target such an operating
system\cite{scharr_ventir}.

\section{Structure, Behavior and Mechanisms}
Ventir is an ordinary 64-bit mach-o executable with multiple mach-o files in the
data section of the executable. Similar to most Trojan-Droppers one of these
mach-o files are set to autorun\cite{erwin_ventir_2014}.

More generally, the keylogger consists of three main components: a trojan
dropper, keylogger and backdoor. The way these components are utilized by the
attacker depends on whether the attacker was able to gain root
access\cite{kuzin_ventir_2014}.

\subsection{Trojan-Dropper.OSX.Ventir.a}
Immediately after the malware is executed, the dropper downloads a set of files
to the victim's machine. The location of these files are dependent on whether
the trojan obtains root access. These files will be dropped into the the
victim's home directory instead of the root directory if root access is
unavailable\cite{kuzin_ventir_2014}. The files that are dropped include the
aforementioned keylogger and backdoor files along with global configuration
files that allow the malware to communicate with the server that controls the
malware. After these files are successfully dropped, the dropper loads the
logging driver into the kernel, if root access is available, otherwise a
Trojan-Spy is downloaded onto the target system. The dropper then removes itself
from the system\cite{erwin_ventir_2014}.

\subsection{Trojan-Spy.OSX.Ventir.a}
The Trojan-Spy or EventMonitor is only downloaded if the dropper was unable to
get root access. It installs its own version of system event handlers to
intercept keystrokes using Carbon Event Manager functions which do not require
root access\cite{kuzin_ventir_2014}.

\subsection{Backdoor.OSX.Ventir.a}
The backdoor file is executed in an infinite loop and continuously communicates
with the command and control server (C\&C). The backdoor can understand and
execute a variety of commands from the server such as reboot/restart the
victim's computer, download/upload/execute files and uninstall itself.

\subsection{not-a-virus:Monitor.OSX.LogKext.c}
Of particular interest is the keylogger file. This file is based on
\textit{logKext}, which is an open source OSX keylogger and can be found on
online code repositories such as Github\cite{logkext}. This file records the
victims keyboard activity into a logfile which can then be sent using the
backdoor malware. However, this file hooks onto the kernel only with elevated
privileges\cite{erwin_ventir_2014}.

\section{Countermeasures \& Detection}

Following the principles of least privilege only trusted programs that require
elevated privileges should recieve it. Because the keylogger component of Ventir
cannot function without elevated privileges, ensuring only trusted programs have
access to root would shut down the effectiveness of the keylogger. In
addition, disabling AutoPlay should prevent the malware from being automatically
executed.

Ventir is easy to detect because it installs all its components into the
\lstinline[basicstyle=\ttfamily]|/Library/.local| and
\lstinline[basicstyle=\ttfamily]|~/Library/.local| directories. Indeed, an
effective eradication scheme for Ventir would just require detection and removal
of these key directories\cite{salmela_roll}.

However, since the backdoor component can download more files and effectively,
more malware, using a firewall to prevent incoming connections while only
allowing trusted services outside access would also be a favorable prevention
method. To be completely safe though, experts recommend performing a total
system reinstall

\section{Security Principles}
By itself, OSX does a decent job of adhering to security principals and thus
preventing the Ventir malware from being as effective. The effectiveness of
Ventir is severely crippled without root access. A security principle OSX could
perhaps enforce even more to reduce the effectiveness of Ventir is the principle
of complete mediation. When a suspicious or untrusted program is executed, OSX
prompts the user for permission. It may make the system even more secure to
prompt the user for permission each time the program is run.

The implementation of the principle of least privilege could also be improved.
When a program requests elevates privileges, the system could show a list of
system modules that the program is trying to access (similar to the prompt shown
when running apps on Android smartphones for the first time). In this way, the
user could see and control the privileges given to programs.

\section{Conclusion}
The Ventir keylogger is an interesting malware to study as it shows that more
malware vendors are choosing to target OSX. In addition, the growth of
open-source software these past few years may be making easier for malware
creators to develop malware. This would concievably lead to an influx of malware
in the future.

\printbibliography

\end{document}
